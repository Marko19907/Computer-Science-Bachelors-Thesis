
\section{Background}

Configuration files are an essential component of software applications, providing a way to store and manage settings and options that determine how the application behaves. Configuration files can be used to specify parameters such as database connections, network settings, application behavior, and other parameters that can be modified without changing the application's code. \\

\noindent
Configuration files can be implemented in various formats, such as XML, YAML, or JSON. JSON (JavaScript Object Notation) has become a popular choice due to its simplicity, flexibility, and compatibility with many programming languages and platforms. \\

\noindent
While configuration files are a useful tool for managing application settings, they can also present challenges. One of the main challenges is managing the complexity and size of configuration files, which can grow rapidly as an application becomes more complex. Additionally, it can be challenging to ensure that configuration files are properly formatted, validated, and secured, to avoid errors, misconfigurations, or security vulnerabilities. \\


\section{Problem statement}

FiiZK is a company in the aquaculture industry that relies on JSON files for configuring its products. As the number of products and customers grows, FiiZK faces challenges in organizing, maintaining, and validating configuration data. The current manual approach for managing JSON configuration files is ineffective and can result in human errors and misconfigurations, leading to potential product failures and decreased customer satisfaction. In addition to that, managing JSON configurations with the manual approach is a boring and frustrating task, leading to a decrease in the employees motivation at work. \\

\noindent
Therefore, FiiZK has tasked us with developing a web application that addresses these challenges by providing a safe, scalable, effective and reliable solution for managing JSON configuration files. The application should allow for creation, maintenance and validation of JSON files, with a user-friendly interface to improve efficiency and reduce the risk of misconfigurations and human errors. The user interface should contain a editor where properties and values can be added and edited in a simple and easy way, and validate the configurations using a JSON schema containing the required properties and corresponding types. If the configuration is not aligned with the requirements of the schema, the user should be given an error message explaining whats wrong. The users input should be formatted according to the JSON syntax and stored in a JSON file, which is ready to be downloaded and deployed whenever the user wants. \\

\section{Objectives and scope}

The main objective of this project was to improve the efficiency and minimize the risk of errors when managing JSON configuration files. To be more specific, we were supposed to create a user-friendly web application that allows the user to create JSON configurations in a simple and effective way, and to give the user a way to check if the configuration is valid according to a JSON schema defining the requirements of the configuration. The app should provide a way to upload a JSON schema and create configurations that will be validated according to the selected schema. \\


\noindent
The scope of the project included the development of a full-stack web application, where we are responsible of both the front-end and back-end of the application, including the user-interface, database solution, data validation logic and so on. Due to the fact that the objectives of the project could be achieved without the need for sensitive or confidential information from the company, we decided to not have a non-disclosure agreement with the client. As a result of this, the goal was to make a more generalized solution to the problem instead of a tailored solution according to their specific needs. \\

\noindent
To achieve these objectives, we utilized modern technologies such as Next.js, NextAuth, Prisma, tRPC, and ChakraUI. These technologies allowed us to build a secure, scalable, and reliable application that met the project requirements. Our focus was on delivering a high-quality application that met the needs of our target users while remaining flexible and adaptable to changing requirements. \\

\section{Report structure}

% We hope you found our project interesting so far.
% // Commented out the above, we can't use this language I think? 
In the following chapters, we'll conduct an in-depth examination of our development process and results. A detailed explanation of the technical details and how we worked collaboratively as a team will be provided. This deep dive will be organized in the following structure: \\

\noindent \textbf{Chapter 2 - Theory:} The theory chapter serves as a foundational introduction to the technical aspects and collaborative approaches of our solution. It covers relevant concepts and theories that underpin our development approach and will provide readers with the necessary background to understand the subsequent chapters.\\

\noindent \textbf{Chapter 3 - Methods:} The methods chapter details the methodology and materials used in the project, explaining the implementation of relevant technologies and concepts. It provides insight into the decision-making process behind the selection of methodologies and materials, offering a comprehensive understanding of the project's execution.\\

\noindent \textbf{Chapter 4 - Results:} The fourth chapter will present the final product of our development process, including the achieved outcomes and the decision-making behind them. It will provide a detailed description of the solution and its technical specifications.\\

\noindent \textbf{Chapter 5 - Discussion:} This chapter provides an analysis and reflection on the results presented in the previous chapter. It examines the strengths and weaknesses of the development process and offers insights for potential future improvements\\

\noindent \textbf{Chapter 6 - Conclusion:} The conclusion summarizes the findings and insights presented in the previous chapters. It provides a concise overview of the project's accomplishments, and outlines possible avenues for future development and improvement.\\

\noindent \textbf{Chapter 7 - Societal Impact:} The final chapter examines the potential economic and social implications of the project. It explores how our solution can contribute to addressing relevant societal issues and how it may benefit different sectors or communities.