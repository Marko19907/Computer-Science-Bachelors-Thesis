
\section{Economic effects}

\noindent
The implementation of our web app has the potential to yield significant economic benefits for organizations. The application allows employees to manage JSON configurations with way more efficiency compared to doing it manually with existing tools, which means that they can get more work done in less time. In addition to that, misconfigurations and human errors can lead to product failures and customer dissatisfaction which can be costly for the company. With our solution for validating the configuration files before they are deployed, the risk of such mistakes is greatly reduced. 

\subsection{Open-source}

The project will be open source and available to everyone. This allows for the use, distribution, and development of commercial software utilizing the project's solutions. By making the project freely available, it will be a contribution to the open-source community by providing example code and documentation that can be utilized and learned from by other developers. Since the solution is free to use, it gives small and medium sized businesses who don't have the same amount of resources as a large company a better chance to keep up with the industry without having to worry about resource limitations. \\

\section{Relation to UN's sustainable development goals}

\noindent
Our solution can contribute to SDG 8: Decent Work and Economic Growth, by enabling organizations to increase the efficiency of their employees. When it comes to the topic of decent work, our solution can help improve morale and satisfaction for employees by making a task that is boring and frustrating easier and more bearable, leaving the employees with more energy and motivation. \cite{unitednations_SGD8}


