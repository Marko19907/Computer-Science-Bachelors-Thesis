
\noindent
I dagens verden med mange mikrotjenester og flerbrukerapplikasjoner, er det økende behov for sikre og skalerbare måter å håndtere data på. Vi har nettopp jobbet med et prosjekt som møter dette behovet ved å lage en full-stack webapplikasjon som lar deg administrere konfigurasjonsfiler for distribuerte datarørledninger. Vi brukte en rekke moderne teknologier som Next.js, NextAuth, Prisma, tRPC og ChakraUI, for å skape en pålitelig og rask løsning som passer for prosjektets krav. \\

\noindent
Next.js, som er basert på React, ble brukt for å bygge front-end av applikasjonen. Den har mange kraftige funksjoner, inkludert server-side rendering, som gir en rask og responsiv brukeropplevelse. NextAuth ble brukt for autentisering, slik at applikasjonen ble trygg og sikker. Prisma håndterte databasen, og tRPC gjorde kommunikasjonen mellom front-end og back-end effektiv. ChakraUI ble brukt for å gi et pent og brukervennlig grensesnitt som gjorde hele opplevelsen bedre. \\

\noindent
Applikasjonen vår hadde som hovedfunksjon å validere JSON-filer med JSON-skjemaer, noe som sikret at dataene som ble brukt av tjenestene våre, var trygge og i samsvar med kundenes behov. Vi gjorde også at det var lett å opprette, vedlikeholde og validere konfigurasjonsfilene, slik at det var mindre sjanse for menneskelige feil eller misconfigurations. \\

\noindent
Resultatet av prosjektet vårt er en pålitelig og skalerbar løsning for administrering av konfigurasjonsfiler i et distribuert datapipeline-miljø. Mens løsningen krever ekstra arbeid før den kan brukes i et produksjonsmiljø, la prosjektet et solid grunnlag for videre utvikling til en mer komplett løsning tilpasset bedriftsstandarder. Resultatene av prosjektet tilbyr verdifull forskningsinnsikt og et pålitelig produkt som kan hjelpe utviklere med å lagre, redigere og validere JSON-konfigurasjonsfilen i skybaserte miljøer. \\